\section{Evaluation}\sectionFrame

\begin{frame}{Evaluationsziele}
    \onslide<1-3> {
        Funktionale Validierung:
        \begin{itemize}
            \item Modulare Architektur erfüllt Anforderungen
            \item Parallelbetrieb von Modellen
            \item Stabile Kommunikation
        \end{itemize}
    }
    \onslide<2-3> {
        Echtzeitfähigkeit:
        \begin{itemize}
            \item Latenz
            \item Durchsatz
            \item Resourcenauslastung
        \end{itemize}
    }
    \onslide<3> {
        Rekonstruktionsqualität:
        \begin{itemize}
            \item Quantiativ (F-Score)
            \item Qualitativ (Vollständigkeit, Detailtreue)
        \end{itemize}
    }
\end{frame}

\begin{frame}{Test- und Evaluationsumgebung}
    \begin{columns}
        \begin{column}{0.5\textwidth}
            \textbf{Hardware (Backend)}
            \begin{itemize}
                \item CPU: AMD Ryzen 5 2600
                \item GPU: NVIDIA GTX 1070 Ti (8GB)
                \item RAM: 16 GB
                \item Docker 29.0.1
            \end{itemize}
        \end{column}

        \begin{column}{0.5\textwidth}
            \textbf{VR-System \& Netzwerk}
            \begin{itemize}
                \item Meta Quest 3
                \item WiFi 5 Heimnetzwerk
                \item Unity 6000.2.12f1
                \item 60 Mbit Internet
            \end{itemize}
        \end{column}
    \end{columns}
\end{frame}

\begin{frame}{Testzenen im Überblick}
    \textbf{3 virtuelle Szenen} (mit Ground-Truth)
    \begin{enumerate}
        \item V1 – Geometrische Primitive: Würfel, Torus, Pyramide | 5×5×5m
        \item V2 – Möbliertes Schlafzimmer: Möbel, Okklusionen | 6×5×3m
        \item V3 – Mehrzweckraum: Große Flächen, viele Details | 10×5×3m
    \end{enumerate}
    \textbf{2 reale Szenen} (für Praktikabilität)
    \begin{enumerate}
        \item R1 – Schlafzimmer: Dachschräge, natürliches Licht
        \item R2 – Wohnzimmer: Große Fensterfront, Glasflächen
    \end{enumerate}
\end{frame}

\begin{frame}{Messsmethoden \& Metriken}
        \begin{columns}
        \begin{column}{0.5\textwidth}
            \textbf{Performance}
            \begin{itemize}
                \item Latenz: Fragment-Versand bis Visualisierung
                \item Durchsatz: Fragmente pro Sekunde
                \item Ressourcen: GPU/CPU-Auslastung
            \end{itemize}
        \end{column}

        \begin{column}{0.5\textwidth}
            \textbf{Qualität}
            \begin{itemize}
                \item Quantiativ: F-Score (Präzision \& Recall)
                \item Schwellwert: 10cm zur GT
                \item Qualitativ: Vollständigkeit, Details, Artefakte
            \end{itemize}
        \end{column}
    \end{columns}
\end{frame}

\begin{frame}{Funktionale Validierung - Ergebnisse}
    \begin{columns}
        \begin{column}{0.5\textwidth}
        \textbf{End-to-End-Kommunikation:}
        \begin{itemize}
            \item 60 Testläufe über 6h
            \item 0 Verbindungsabbrüche
        \end{itemize}
        \vspace{1cm}
        \textbf{Parallele Modellausführung:}
        \begin{itemize}
            \item Alle 4 Modelle stabil gleichzeitig
            \item Containerisierung funktioniert
        \end{itemize}    
        \end{column}
        \begin{column}{0.5\textwidth}
        \textbf{Multi-Szenen-Unterstützung:}
        \begin{itemize}
            \item 2 Szenen mit 4 Clients
            \item Fehlerfreie Zuordnung
        \end{itemize}
        \vspace{1cm}
        \textbf{VR-Integration:}
        \begin{itemize}
            \item Meshes und Punktwolken visualisiert
            \item Bis 100k Punkte darstellbar
        \end{itemize}
    \end{column}
    \end{columns}
\end{frame}

\begin{frame}{Latenz - Gesamtübersicht}
    
\end{frame}

\begin{frame}{Latenzkomposition}
    
\end{frame}

\begin{frame}{Fragment- \& Ergebnisgrößen}
    
\end{frame}

\begin{frame}{Durchsatz \& Resourcenauslastung}
    
\end{frame}

\begin{frame}{Rekonstruktionsqualität - F-Score}
    
\end{frame}

\begin{frame}{Qualitative Bewertung}
    
\end{frame}

\begin{frame}{Echtzeitfähigkeit - Zusammenfassung}
    
\end{frame}

\begin{frame}{Diskussion der Ergebnisse}
    
\end{frame}

\begin{frame}{Limitationen \& Beantwortung der Forschungsfrage}
    
\end{frame}