\section{Evaluation}\sectionFrame

\begin{frame}{Evaluationsziele}
    \textbf{1. Funktionale Validierung:}
    \begin{itemize}
        \item Modulare Architektur erfüllt Anforderungen
    \end{itemize}
    \textbf{2. Echtzeitfähigkeit:}
    \begin{itemize}
        \item Performance-Metriken für VR-Szenarien
    \end{itemize}
    \textbf{3. Rekonstruktionsqualität:}
    \begin{itemize}
        \item Modellvergleich unter identischen Bedingungen
    \end{itemize}
    \textbf{4. Systemstabilität:}
    \begin{itemize}
        \item Robustheit unter kontinuierlicher Last
    \end{itemize}
\end{frame}

\begin{frame}{Evaluationsumgebung}
    \begin{columns}
        \column{0.5\textwidth}
            \begin{tcolorbox}[colback=blue!10, colframe=blue!70, boxsep=0.5em, height=3cm, title=\textbf{Backend-Hardware}]
                \begin{itemize}
                    \item CPU: AMD Ryzen 5 5600X
                    \item GPU: NVIDIA GeForce GTX 1070 Ti (8GB VRAM)
                    \item Speicher: 16 GB RAM
                \end{itemize}
            \end{tcolorbox}
        \column{0.5\textwidth}
            \begin{tcolorbox}[colback=teal!10, colframe=teal!70, boxsep=0.5em, height=3cm, title=\textbf{Frontend-Hardware}]
                \begin{itemize}
                    \item VR-Headset: Meta Quest 3
                    \item Controller: Standard Meta Quest Controller
                \end{itemize}
            \end{tcolorbox}
    \end{columns}
    
    \vfill

    \begin{tcolorbox}[colback=orange!10, colframe=orange!70, boxsep=0.5em, title=\textbf{Netzwerk}]
        \begin{itemize}
            \item WiFi 5 (802.11ac)
            \item Durchschnittliche Bandbreite: 60 Mbit/s
        \end{itemize}
    \end{tcolorbox}
\end{frame}

\begin{frame}{Testszenarien}
    \textbf{Virtuelle Szenen (mit Ground-Truth):}
    \begin{itemize}
        \item V1 – Primitive: 5×5×5m, einfache Geometrie
        \item V2 – Schlafzimmer: 6×5×3m, moderate Komplexität
        \item V3 – Mehrzweckraum: 10×5×3m, hohe Komplexität
    \end{itemize}
    \textbf{Reale Szenen:}
    \begin{itemize}
        \item R1 – Schlafzimmer: Kontraste, Details, Okklusionen
        \item R2 – Wohnzimmer: Glas, Reflexionen, große Flächen
    \end{itemize}
    \textbf{Fenstergröße:} 9 Frames pro Fragment
\end{frame}

\begin{frame}{Funktionale Validierung - Ergebnisse}
    \textbf{1. End-to-End-Kommunikation:}
    \begin{itemize}
        \item 60 Testläufe (6h), 0 Verbindungsabbrüche
    \end{itemize}
    \textbf{2. Parallele Modellausführung:}
    \begin{itemize}
        \item Alle 4 Modelle konfliktfrei nebeneinander
    \end{itemize}
    \textbf{3. Multi-Szenen-Support:}
    \begin{itemize}
        \item Fehlerfrei mit 2 Szenen, 4 Clients
    \end{itemize}
    \textbf{4. VR-Visualisierung:}
    \begin{itemize}
        \item Meshes und Punktwolken (bis 100k Punkte) dargestellt
    \end{itemize}
\end{frame}

\begin{frame}{Performance-Analyse: Gesamtlatenz}
    \begin{itemize}
        \item Stabile, reproduzierbare Latenzwerte mit niedriger Varianz
        \item Trend: Komplexität $\uparrow$ $\longrightarrow$ Latenz $\uparrow$
        \item SLAM-Verfahren zeigen robusteres Verhalten in realen Szenen
        \item Starke szenenabhängige Unterschiede beobachtet
    \end{itemize}

    \vfill

    \begin{figure}
        \begin{subfigure}{0.33\textwidth}
            \includegraphics[width=\textwidth]{images/room00_latency.png}
            \caption{V1}
        \end{subfigure}
        \begin{subfigure}{0.33\textwidth}
            \includegraphics[width=\textwidth]{images/room02_latency.png}
            \caption{V3}
        \end{subfigure}
        \begin{subfigure}{0.33\textwidth}
            \includegraphics[width=\textwidth]{images/room04_latency.png}
            \caption{R1}
        \end{subfigure}
    \end{figure}
\end{frame}

\begin{frame}{Performance-Analyse: Latenz-Zusammensetzung}
    \begin{itemize}
        \item \textbf{$L_{inference}$:} Dominierender Anteil (variiert stark mit Szenenkomplexität)
        \item \textbf{$L_{network}$ und $L_{render}$:} Relativ konstant, prozentual kleiner bei komplexeren Szenen
        \item Netzwerklatenz abhängig von Datengröße und verfügbarer Bandbreite
    \end{itemize}
    
    \begin{figure}
        \includegraphics[width=\textwidth]{images/latency_split.png}
    \end{figure}
\end{frame}

\begin{frame}{Performance-Analyse: Durchsatz} 
    \begin{table}[h]
    \centering
    \resizebox{0.7\textwidth}{!}{
        \begin{tabular}{l|ccccc}
        \textbf{Modell} & \textbf{V1} & \textbf{V2} & \textbf{V3} & \textbf{R1} & \textbf{R2} \\
        \hline
        NeuralRecon & 0,62 & 0,45 & 0,32 & 0,48 & 0,45 \\
        VisFusion & 0,38 & 0,35 & 0,30 & 0,38 & 0,31 \\
        SLAM3R & 0,13 & 0,12 & 0,12 & 0,12 & 0,12 \\
        MASt3R-SLAM & 0,07 & 0,06 & 0,03 & 0,10 & 0,09
        \end{tabular}
    }
    \end{table}
    
    \vspace{1em}

    \textbf{Beobachtungen:}
    \begin{itemize}
        \item Volumetrische Verfahren deutlich schneller
        \item SLAM-Verfahren rechenintensiver durch Feature-Matching
        \item Reale Szenen teilweise performanter als virtuelle
    \end{itemize}
\end{frame}

\begin{frame}{Rekonstruktionsqualität: Quantitative Ergebnisse}    
    \begin{table}[h]
    \centering
    \resizebox{\textwidth}{!}{
    \begin{tabular}{l|ccc|ccc|ccc}
    \textbf{Modell} & \multicolumn{3}{c|}{\textbf{V1}} & \multicolumn{3}{c|}{\textbf{V2}} & \multicolumn{3}{c}{\textbf{V3}} \\
    & Prec. & Rec. & F & Prec. & Rec. & F & Prec. & Rec. & F \\
    \hline
    NeuralRecon & 0,45 & 0,39 & 0,41 & 0,69 & 0,67 & \textbf{0,68} & 0,59 & 0,59 & 0,59 \\
    VisFusion & 0,57 & 0,52 & 0,54 & 0,57 & 0,65 & 0,61 & 0,57 & 0,68 & \textbf{0,62} \\
    SLAM3R & \textbf{0,66} & 0,56 & \textbf{0,61} & 0,67 & 0,59 & 0,63 & 0,41 & 0,43 & 0,42 \\
    MASt3R-SLAM & 0,52 & 0,48 & 0,50 & 0,47 & 0,53 & 0,50 & 0,43 & 0,48 & 0,45
    \end{tabular}
    }
    \caption{F-Score, Schwellwert von 10cm}
    \end{table}

    \vspace{1em}

    \textbf{Szenenabhängige Leistung:}
    \begin{itemize}
        \item V2 (Schlafzimmer) zeigt beste Ergebnisse für NeuralRecon
        \item V3 (Mehrzweckraum) mehr Variabilität zwischen Verfahren
    \end{itemize}
\end{frame}

\begin{frame}{Rekonstruktionsqualität: Qualitative Bewertung}
    \textbf{Bewertungskriterien:}
    \begin{itemize}
        \item Vollständigkeit: Erfassungsrate der Szene
        \item Detailtreue: Feine Strukturen und Kanten
        \item Artefaktfreiheit: Löcher, Flimmern, Fehlgeometrie
        \item Oberflächenqualität: Glattheit und Konsistenz
    \end{itemize}
    
    \vfill
    
    \begin{table}[h]
    \centering
    \resizebox{0.95\textwidth}{!}{
    \begin{tabular}{l|cccc}
    \textbf{Kriterium} & \textbf{NeuralRecon} & \textbf{VisFusion} & \textbf{SLAM3R} & \textbf{MASt3R} \\
    \hline
    Vollständigkeit & mittel & hoch & sehr hoch & hoch \\
    Detailtreue & gering & mittel & sehr hoch & sehr hoch \\
    Artefaktfreiheit & sehr hoch & sehr hoch & niedrig & niedrig \\
    Oberflächenqualität & sehr hoch & sehr hoch & mittel & mittel
    \end{tabular}
    }
    \end{table}
\end{frame}
