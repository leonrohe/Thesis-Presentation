\section{Evaluation}\sectionFrame

\begin{frame}{Evaluationsziele}
    \onslide<1-3> {
        Funktionale Validierung:
        \begin{itemize}
            \item Modulare Architektur erfüllt Anforderungen
            \item Parallelbetrieb von Modellen
            \item Stabile Kommunikation
        \end{itemize}
    }
    \onslide<2-3> {
        Echtzeitfähigkeit:
        \begin{itemize}
            \item Latenz
            \item Durchsatz
            \item Resourcenauslastung
        \end{itemize}
    }
    \onslide<3> {
        Rekonstruktionsqualität:
        \begin{itemize}
            \item Quantiativ (F-Score)
            \item Qualitativ (Vollständigkeit, Detailtreue)
        \end{itemize}
    }
\end{frame}

\begin{frame}{Test- und Evaluationsumgebung}
    \begin{columns}
        \begin{column}{0.5\textwidth}
            \textbf{Hardware (Backend)}
            \begin{itemize}
                \item CPU: AMD Ryzen 5 2600
                \item GPU: NVIDIA GTX 1070 Ti (8GB)
                \item RAM: 16 GB
                \item Docker 29.0.1
            \end{itemize}
        \end{column}

        \begin{column}{0.5\textwidth}
            \textbf{VR-System \& Netzwerk}
            \begin{itemize}
                \item Meta Quest 3
                \item WiFi 5 Heimnetzwerk
                \item Unity 6000.2.12f1
                \item 60 Mbit Internet
            \end{itemize}
        \end{column}
    \end{columns}
\end{frame}

\begin{frame}{Testzenen im Überblick}
    \textbf{3 virtuelle Szenen} (mit Ground-Truth)
    \begin{enumerate}
        \item V1 – Geometrische Primitive: Würfel, Torus, Pyramide | 5×5×5m
        \item V2 – Möbliertes Schlafzimmer: Möbel, Okklusionen | 6×5×3m
        \item V3 – Mehrzweckraum: Große Flächen, viele Details | 10×5×3m
    \end{enumerate}
    \textbf{2 reale Szenen} (für Praktikabilität)
    \begin{enumerate}
        \item R1 – Schlafzimmer: Dachschräge, natürliches Licht
        \item R2 – Wohnzimmer: Große Fensterfront, Glasflächen
    \end{enumerate}
\end{frame}

\begin{frame}{Messsmethoden \& Metriken}
        \begin{columns}
        \begin{column}{0.5\textwidth}
            \textbf{Performance}
            \begin{itemize}
                \item Latenz: Fragment-Versand bis Visualisierung
                \item Durchsatz: Fragmente pro Sekunde
                \item Ressourcen: GPU/CPU-Auslastung
            \end{itemize}
        \end{column}

        \begin{column}{0.5\textwidth}
            \textbf{Qualität}
            \begin{itemize}
                \item Quantiativ: F-Score (Präzision \& Recall)
                \item Schwellwert: 10cm zur GT
                \item Qualitativ: Vollständigkeit, Details, Artefakte
            \end{itemize}
        \end{column}
    \end{columns}
\end{frame}

\begin{frame}{Funktionale Validierung}
    \begin{columns}
        \begin{column}{0.5\textwidth}
        \textbf{End-to-End-Kommunikation:}
        \begin{itemize}
            \item 60 Testläufe über 6h
            \item 0 Verbindungsabbrüche
        \end{itemize}
        \vspace{1cm}
        \textbf{Parallele Modellausführung:}
        \begin{itemize}
            \item Alle 4 Modelle stabil gleichzeitig
            \item Containerisierung funktioniert
        \end{itemize}    
        \end{column}
        \begin{column}{0.5\textwidth}
        \textbf{Multi-Szenen-Unterstützung:}
        \begin{itemize}
            \item 2 Szenen mit 4 Clients
            \item Fehlerfreie Zuordnung
        \end{itemize}
        \vspace{1cm}
        \textbf{VR-Integration:}
        \begin{itemize}
            \item Meshes und Punktwolken visualisiert
            \item Bis 100k Punkte darstellbar
        \end{itemize}
    \end{column}
    \end{columns}
\end{frame}

\begin{frame}{End-To-End-Latenz - Definition}
    
    {
        \Large
        \begin{align*}
            L_{\text{total}} = L_{\text{network}} + L_{\text{inference}} + L_{\text{render}} \\
        \end{align*}
    }  

    \begin{itemize}
        \item $L_{\text{network}}$~:~Benötigte Zeit zum Up- und Download des Fragments
        \item $L_{\text{inference}}$~:~Benötigte Zeit für die Inferrenz des Fragments
        \item $L_{\text{render}}$~:~Benötigte Zeit für das Darstellen des Modell Ergebnisses
    \end{itemize}
\end{frame}

\begin{frame}{End-To-End-Latenz - Ergebnisse - V1}
    \centering
    \includegraphics{images/room00_latency.png}
\end{frame}

\begin{frame}{End-To-End-Latenz - Ergebnisse - V2}
    \centering
    \includegraphics{images/room01_latency.png}
\end{frame}

\begin{frame}{End-To-End-Latenz - Ergebnisse - V3}
    \centering
    \includegraphics{images/room02_latency.png}
\end{frame}

\begin{frame}{End-To-End-Latenz - Ergebnisse - R1}
    \centering
    \includegraphics{images/room03_latency.png}
\end{frame}

\begin{frame}{End-To-End-Latenz - Ergebnisse - R2}
    \centering
    \includegraphics{images/room04_latency.png}
\end{frame}

\begin{frame}{End-To-End-Latenz - Komposition}
    \centering
    \includegraphics[width=\textwidth]{images/latency_split.png}
\end{frame}

\begin{frame}{Fragment- \& Ergebnisgrößen - Virtuelle Szenen}
    \begin{table}[h]
        \centering
        \label{tab:data_volumes}
        \resizebox{\textwidth}{!}{%
        \begin{tabular}{l|l|c|c|c|c}
            \toprule
            \textbf{Szene} & \textbf{Modell} & \makecell{\textbf{\O~$\text{Frag}_{\text{in}}$~[MB]}} & \makecell{\textbf{\O~$\text{Frag}_{\text{out}}$~[MB]}} & \makecell{\textbf{$\sum\text{Frag}_{\text{in}}$~[MB]}} & \makecell{\textbf{$\sum\text{Frag}_{\text{out}}$~[MB]}} \\
            \midrule
            \multirow{4}{*}{V1} 
                & NeuralRecon  & \multirow{4}{*}{0{,}82} & 1{,}83 & \multirow{4}{*}{19{,}57} & 44{,}04 \\
                & VisFusion    &                       & 2{,}73 &                        & 65{,}40 \\
                & MASt3R-SLAM  &                       & 1{,}60 &                        & 38{,}42 \\
                & SLAM3R       &                       & 1{,}60 &                        & 38{,}42 \\
            \midrule
            \multirow{4}{*}{V2} 
                & NeuralRecon  & \multirow{4}{*}{1{,}62} & 2{,}84 & \multirow{4}{*}{69{,}57} & 122{,}28 \\
                & VisFusion    &                      & 3{,}25  &                        & 139{,}85 \\
                & MASt3R-SLAM  &                      & 1{,}60  &                        & 68{,}60 \\
                & SLAM3R       &                      & 1{,}60  &                        & 68{,}84 \\
            \midrule
            \multirow{4}{*}{V3} 
                & NeuralRecon  & \multirow{4}{*}{1{,}56} & 3{,}60 & \multirow{4}{*}{88{,}70} & 204{,}93 \\
                & VisFusion    &                      & 4{,}24  &                        & 241{,}51 \\
                & MASt3R-SLAM  &                      & 1{,}59  &                        & 90{,}69 \\
                & SLAM3R       &                      & 1{,}60  &                        & 91{,}26 \\
            \bottomrule
        \end{tabular}%
        }
    \end{table}
\end{frame}

\begin{frame}{Fragment- \& Ergebnisgrößen - Reale Szenen}
    \begin{table}[h]
        \centering
        \label{tab:data_volumes}
        \resizebox{\textwidth}{!}{%
        \begin{tabular}{l|l|c|c|c|c}
            \toprule
            \textbf{Szene} & \textbf{Modell} & \makecell{\textbf{\O~$\text{Frag}_{\text{in}}$~[MB]}} & \makecell{\textbf{\O~$\text{Frag}_{\text{out}}$~[MB]}} & \makecell{\textbf{$\sum\text{Frag}_{\text{in}}$~[MB]}} & \makecell{\textbf{$\sum\text{Frag}_{\text{out}}$~[MB]}} \\
            \midrule
            \multirow{4}{*}{R1} 
                & NeuralRecon  & \multirow{4}{*}{1{,}63} & 2{,}19 & \multirow{4}{*}{78{,}50} & 105{,}29 \\
                & VisFusion    &                       & 2{,}58 &                       & 124{,}14 \\
                & MASt3R-SLAM  &                       & 1{,}60 &                       & 76{,}85 \\
                & SLAM3R       &                       & 1{,}60 &                       & 76{,}85 \\
            \midrule
            \multirow{4}{*}{R2} 
                & NeuralRecon  & \multirow{4}{*}{1{,}84} & 2{,}48 & \multirow{4}{*}{79{,}33} & 106{,}87 \\
                & VisFusion    &                       & 3{,}70 &                        & 159{,}19 \\
                & MASt3R-SLAM  &                       & 1{,}60 &                        & 68{,}84 \\
                & SLAM3R       &                       & 1{,}60 &                        & 68{,}84 \\
            \bottomrule
        \end{tabular}%
        }
    \end{table}
\end{frame}

\begin{frame}{Durchsatz}
    \textbf{Volumetrische Verfahren}
    \begin{itemize}
        \item NeuralRecon: 0,32 - 0,62 Fragmente/s
        \item VisFusion: 0,30 - 0,38 Fragmente/s
    \end{itemize}

    \textbf{SLAM Verfahren}
    \begin{itemize}
        \item MaSt3R-SLAM: 0,12 - 0,13 Fragmente/s
        \item SLAM3R: 0,03 - 0,10 Fragmente/s
    \end{itemize}

    \textbf{Trade-Off:} Schneller aber weniger reobust (volumetrisch) vs. Langsam, aber robuster (SLAM)
\end{frame}

\begin{frame}{Ressourcenauslastung}
        \begin{columns}
        \begin{column}{0.5\textwidth}
            \textbf{GPU-Auslastung}
            \begin{itemize}
                \item Latenz: Fragment-Versand bis Visualisierung
                \item Durchsatz: Fragmente pro Sekunde
                \item Ressourcen: GPU/CPU-Auslastung
            \end{itemize}
        \end{column}

        \begin{column}{0.5\textwidth}
            \textbf{Qualität}
            \begin{itemize}
                \item Quantiativ: F-Score (Präzision \& Recall)
                \item Schwellwert: 10cm zur GT
                \item Qualitativ: Vollständigkeit, Details, Artefakte
            \end{itemize}
        \end{column}
    \end{columns}
\end{frame}

\begin{frame}{Rekonstruktionsqualität - F-Score}
    
\end{frame}

\begin{frame}{Qualitative Bewertung}
    
\end{frame}

\begin{frame}{Echtzeitfähigkeit - Zusammenfassung}
    
\end{frame}

\begin{frame}{Diskussion der Ergebnisse}
    
\end{frame}

\begin{frame}{Limitationen \& Beantwortung der Forschungsfrage}
    
\end{frame}