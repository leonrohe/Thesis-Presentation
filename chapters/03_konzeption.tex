\section{Konzeption}\sectionFrame

\begin{frame}{Funktionale Anforderungen}
    \begin{itemize}
        \item Kontinuierliche Datenerfassung: RGB-Bilder + Posen zeitlich fensterbasiert
        \item Fensterbasierte Übertragung: Fragmente eindeutig zu Szene \& Modell
        \item Rekonstruktion durch mehrere Modelle: Parallel NeuralRecon, VisFusion, MASt3R-SLAM, SLAM3R
        \item Rückführung der Ergebnisse: 3D-Geometrie an VR-Clients
        \item Szenen- \& Mehrbenutzerverwaltung: Mehrere Clients in einer Szene
        \item Persistenz der Rekonstruktionen: Backend-Speicherung für Evaluation
    \end{itemize}
\end{frame}

\begin{frame}{Nicht-funktionale Anforderungen}
    \begin{itemize}
        \item Niedrige Latenz: VR-tauglich (HMD-Ausgabe akzeptabel)
        \item Stabilität: Paketverlust, ausfallende Modelle tolerierbar
        \item Skalierbarkeit: Mehrere parallele Worker ohne Anpassungen
        \item Modularität der Modelle: Entkoppelte Services mit einheitlicher Schnittstelle
        \item Wiederverwendbarkeit: Generische Infrastruktur für verschiedene Szenarien
    \end{itemize}
\end{frame}

\begin{frame}{Gesamtarchitektur}
    \onslide<1-3> {
        Router (Backend):
        \begin{itemize}
            \item Aufgabe: Vermittlung und Verteilung
        \end{itemize}
    }
    \onslide<2-3> {
        Worker (containerisiert):
        \begin{itemize}
            \item Erfassung und Visualisierung
        \end{itemize}
    }
    \onslide<3> {
        Frontend (Unity):
        \begin{itemize}
            \item Aufgabe: Kapselung der Rekonstruktionsmodelle
        \end{itemize}
    }
\end{frame}