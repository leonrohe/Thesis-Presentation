\section{Konzeption}\sectionFrame

\begin{frame}{Anforderungen an RTReconstruct}
    \begin{columns}[T]
        \column{0.5\textwidth}
        {\large \textbf{Funktional}}
        \begin{itemize}
            \itemsep0.8em
            \item Echtzeitverarbeitung
            \item Kontinuierliche Datenerfassung
            \item Fensterbasierte Übertragung
        \end{itemize}
        
        \column{0.5\textwidth}
        {\large \textbf{Nicht-funktional}}
        \begin{itemize}
            \itemsep0.8em
            \item Niedrige Latenz
            \item Modularität
            \item Stabilität \& Skalierbarkeit
        \end{itemize}
    \end{columns}
    \vfill
    \begin{tcolorbox}[colback=yellow!5, colframe=gray!70, boxsep=0.5em]
        \textbf{Ziel:} Heterogene Rekonstruktionsverfahren in einer gemeinsamen Architektur
    \end{tcolorbox}
\end{frame}

\begin{frame}{Gesamtarchitektur}
    \centering
    \includesvg[width=0.9\textwidth]{figures/export/gesamtarchitektur}
\end{frame}

\begin{frame}{Backend-Architektur: Router \& Worker}
    \begin{columns}[T]
        \column{0.5\textwidth}
        {\large \textbf{Router}}
        \begin{itemize}
            \itemsep0.6em
            \item Empfängt Fragmente
            \item Zuordnung zu Szenen
            \item Caching aktueller Ergebnisse
            \item Fan-Out-Verteilung
        \end{itemize}
        
        \column{0.5\textwidth}
        {\large \textbf{Worker}}
        \begin{itemize}
            \itemsep0.6em
            \item Eigenständige Dienste
            \item Modellspezifische Logik
            \item Async. Kommunikation
            \item GPU-gekapselt in Container
        \end{itemize}
    \end{columns}
    
    \vfill
    \begin{tcolorbox}[colback=yellow!5, colframe=gray!70, boxsep=0.5em]
        \textbf{Entkoppelung:} Asynchrone Pufferung ermöglicht unabhängige Verarbeitung
    \end{tcolorbox}
\end{frame}

\begin{frame}{Frontend-Architektur: Drei Aufgabenbereiche}
    \begin{tcolorbox}[colback=blue!10, colframe=blue!70, boxsep=0.5em, title=\textbf{1. Erfassung und Vorverarbeitung}]
        \begin{itemize}
            \itemsep0.3em
            \small
            \item XR-Endgeräte (HMD, mobile Kameras)
            \item RGB-Video + kontinuierliche Pose-Daten
        \end{itemize}
    \end{tcolorbox}
    
    \vspace{0.6em}
    
    \begin{tcolorbox}[colback=teal!10, colframe=teal!70, boxsep=0.5em, title=\textbf{2. Clientseitige Steuerung \& Fragmentbildung}]
        \begin{itemize}
            \itemsep0.3em
            \small
            \item Modellverwaltung pro Szene
            \item Generische + modellspezifische Logik
        \end{itemize}
    \end{tcolorbox}
    
    \vspace{0.6em}
    
    \begin{tcolorbox}[colback=orange!10, colframe=orange!70, boxsep=0.5em, title=\textbf{3. Visualisierung \& Va.Si.Li-Lab-Integration}]
        \begin{itemize}
            \itemsep0.3em
            \small
            \item Einbettung in laufende VR-Szene
            \item Keine Anpassung der Basis-Szenlogik
        \end{itemize}
    \end{tcolorbox}
\end{frame}

\begin{frame}{Modul- und Schnittstellendesign}    
    \begin{columns}[T]
        \column{0.33\textwidth}
        \begin{tcolorbox}[colback=blue!10, colframe=blue!70, boxsep=0.5em, height=3.5cm, title=\textbf{Anfrage}]
            \small Enthält:
            \begin{itemize}
                \itemsep0.2em
                \item Szene-ID
                \item Modell-ID
                \item Zeitfenster
                \item Kamerageometrie
            \end{itemize}
        \end{tcolorbox}
        
        \column{0.33\textwidth}
        \begin{tcolorbox}[colback=teal!10, colframe=teal!70, boxsep=0.5em, height=3.5cm, title=\textbf{Antwort}]
            \small Enthält:
            \begin{itemize}
                \itemsep0.2em
                \item Szene-ID
                \item Modell-ID
                \item Repräsentation
                \item Transformation
            \end{itemize}
        \end{tcolorbox}
        
        \column{0.33\textwidth}
        \begin{tcolorbox}[colback=orange!10, colframe=orange!70, boxsep=0.5em, height=3.5cm, title=\textbf{Verwaltung}]
            \small Enthält:
            \begin{itemize}
                \itemsep0.2em
                \item Lifecycle
                \item Konfiguration
                \item Events
            \end{itemize}
        \end{tcolorbox}
    \end{columns}
    
    \vfill
    \begin{tcolorbox}[colback=yellow!5, colframe=gray!70, boxsep=0.5em]
        \centering
        \textbf{Vorteil:} Vermittlungsschicht unabhängig von Modellimplementierungen
    \end{tcolorbox}
\end{frame}

\begin{frame}{Kommunikations- und Datenfluss}    
    \begin{columns}[T]
        \column{0.25\textwidth}
        \begin{tcolorbox}[colback=blue!10, colframe=blue!70, height=2.75cm, boxsep=0.2em]
            \centering
            \textcolor{blue!70}{\textbf{[1]}}\\
            \textbf{Verbindung}\\
            \vspace{0.2em}
            \small WebSocket-Handshake\\
            Frontend $\to$ Router\\
            (Modellabfrage)
        \end{tcolorbox}
        
        \column{0.25\textwidth}
        \begin{tcolorbox}[colback=teal!10, colframe=teal!70, height=2.75cm, boxsep=0.2em]
            \centering
            \textcolor{teal!70}{\textbf{[2]}}\\
            \textbf{Erfassung}\\
            \vspace{0.2em}
            \small Bilder + Posen\\
            sammeln und\\
            zu Fragmenten\\
            bündeln
        \end{tcolorbox}
        
        \column{0.25\textwidth}
        \begin{tcolorbox}[colback=orange!10, colframe=orange!70, height=2.75cm, boxsep=0.2em]
            \centering
            \textcolor{orange!70}{\textbf{[3]}}\\
            \textbf{Rekonstruktion}\\
            \vspace{0.2em}
            \small Fragment an\\
            Worker-Queue\\
            3D-Rekonstruktion\\
            durchfuehren
        \end{tcolorbox}
        
        \column{0.25\textwidth}
        \begin{tcolorbox}[colback=red!10, colframe=red!70, height=2.75cm, boxsep=0.2em]
            \centering
            \textcolor{red!70}{\textbf{[4]}}\\
            \textbf{Visualisierung}\\
            \vspace{0.2em}
            \small Ergebnis speichern,\\
            Fan-Out an\\
            alle Clients,\\
            Visualisierung
        \end{tcolorbox}
    \end{columns}
    
    \vfill
    \begin{tcolorbox}[colback=yellow!5, colframe=gray!70, boxsep=0.5em]
        \small \textbf{Asynchrone Entkopplung:} Jede Phase arbeitet mit eigenem Tempo durch Pufferung
    \end{tcolorbox}
\end{frame}