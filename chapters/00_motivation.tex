\section{Motivation}\sectionFrame

\begin{frame}{Virtual-Reality und die Realität}
    \begin{itemize}
        \item Moderne VR-Systeme ermöglichen natürliche Interaktion durch präzises Tracking
        \item Passthrough-Funktionen erlauben Mixed-Reality-Anwendungen
        \item Problem: Viele VR-Erfahrungen basieren auf künstlich erstellten Szenen
        \item Nachteil für spezifische Anwendungsfelder:
        \begin{itemize}
            \item Kollaboratives Arbeiten
            \item Simulationsbasiertes Lernen
            \item Räumliche Analysewerkzeuge
        \end{itemize}
        \item Diese Anwendungen benötigen enge Beziehung zwischen digitaler und realer Welt
    \end{itemize}
\end{frame}

\begin{frame}{Warum 3D-Rekonstruktion}
    \begin{itemize}
        \item Virtuelle Räume wirken glaubwürdiger, wenn sie aus realen Szenen abgeleitet sind
        \item Praktische Vorteile:
        \begin{itemize}
            \item Spontane Erfassung von Umgebungen
            \item Gemeinsame Analyse und Visualisierung
            \item Kollaborative Aufgaben ohne aufwändige manuelle Modellierung
        \end{itemize}
        \item Zentrale Komponente für nächste Generation interaktiver VR-Systeme
        \item Erspart zeit- und kostenintensive Modellierungsprozesse
    \end{itemize}
\end{frame}

\begin{frame}{Von Photogrammetrie zu Echtzeit-Verfahren}
    Klassische Photogrammetrie:
    \begin{itemize}
        \item Hochauflösende Resultate
        \item Umfangreiche Bildaufnahmen erforderlich
        \item Lange Verarbeitungszeiten
        \item[$\Longrightarrow$] Ungeeignet für dynamische/interaktive VR-Anwendungen
    \end{itemize}
    \onslide<2>{
        Monokulare-Echtzeit-Verfahren:
        \begin{itemize}
            \item Fortlaufende RGB-Videodaten nutzen
            \item Geometrie schrittweise rekonstruieren
            \item Direkt auf VR-Headset-Kameras angewendet
            \item Keine zusätzliche Hardware erforderlich
            \item[$\Longrightarrow$] Gute Eignung für dynamische/interaktive VR-Anwendungen
        \end{itemize}
    }
\end{frame}

\begin{frame}{Das Integrationsproblem}
    Aktuelle Forschungsprototypen:
    \begin{itemize}
        \item Entwickelt in isolierten Forschungskontexten
        \item Eigene Laufzeitumgebungen und Frameworks
        \item Heterogene Datenpipelines
    \end{itemize}
    \onslide<2> {
        Anforderungen von VR-Anwendungen:
        \begin{itemize}
            \item Stabiler, strukturierter Datenfluss
            \item Zuverlässige Bereitstellung kontinuierlich wachsender Modelle
            \item Nahtlose Frontend-Integration
            \item Forschungslücke: Modulare Architektur zur Kapselung verschiedener Verfahren bislang nur unzureichend adressiert
        \end{itemize}
    }
\end{frame}

\begin{frame}{Forschungsfrage und Ziel}
    \only<2> {
        \begin{center}
            \Large
            \textit{Wie gut eignet sich eine modulare, containerisierte Systemarchitektur zur Integration verschiedener Echtzeit-3D-Rekonstruktionsverfahren in eine bestehende Virtual-Reality-Umgebung?}
        \end{center}
    }
\end{frame}