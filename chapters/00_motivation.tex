\section{Motivation}\sectionFrame

\begin{frame}{3D-Rekonstruktion in VR}{(\cite{enhancinginteractions,Stotko_2019,khenak2020spatial})}
    \textbf{Ausgangssituation:}
    \begin{itemize}
        \item VR-Anwendungen basieren oft auf künstlich erstellten Szenen
        \item Viele Anwendungsfälle profitieren von Bezug zur realen Welt
    \end{itemize}
    
    \bigskip
    
    \textbf{Vorteile durch Rekonstruktion:}
    \begin{itemize}
        \item Spontane Erfassung und gemeinsame Analyse von Räumen
        \item Kollaborative Aufgaben ohne aufwendige manuelle Modellierung
        \item Grundbaustein für dynamische, interaktive VR-Systeme
    \end{itemize}
\end{frame}

\begin{frame}{Klassische Photogrammetrie: Ungeeignet}{(\cite{Day2025RealTimePhotogrammetry})}
    \begin{columns}[T]
        \column{0.5\textwidth}
        \textbf{Vorteile:}
        \begin{itemize}
            \item Hochauflösende Resultate
        \end{itemize}
        
        \column{0.5\textwidth}
        \textbf{Nachteile:}
        \begin{itemize}
            \item Umfangreiche Bildaufnahmen
            \item Lange Verarbeitungszeiten
            \item Ungeeignet für dynamische VR
        \end{itemize}
    \end{columns}
\end{frame}

\begin{frame}{Monokulare Echtzeit-Rekonstruktion: Potenzial \& Herausforderung}{(\cite{sun2021neuralreconrealtimecoherent3d,gao2023visfusionvisibilityawareonline3d,liu2025slam3rrealtimedensescene, murai2025mast3rslamrealtimedenseslam})}
    \textbf{Vorteile:}
    \begin{itemize}
        \item Keine zusätzliche Hardware erforderlich
        \item Direkte Nutzung moderner VR-Headset-Kameras
    \end{itemize}
    
    \bigskip
    
    \textbf{Aber: Praktische Integration kaum etabliert}
    \begin{itemize}
        \item Modelle in isolierten Forschungskontexten entwickelt
        \item Heterogene Laufzeitumgebungen und Frameworks
        \item Nicht auf interaktive Systeme ausgelegt
    \end{itemize}
\end{frame}

\begin{frame}{Anforderungen an VR-Rekonstruktionssysteme}
    \textbf{Was die Integration erschwert:}
    \begin{itemize}
        \item \textbf{Stabiler Datenfluss:} Kontinuierliche Bilder + Posen ohne Ausfälle
        \item \textbf{Wachsende Modelle:} Szene wird inkrementell aktualisiert, nicht am Ende
        \item \textbf{Nahtlose Integration:} Muss mit bestehender VR-Engine (z.B.\ Va.Si.Li-Lab) funktionieren
        \item \textbf{Vergleichbarer Betrieb:} Mehrere Rekonstruktionsverfahren parallel testen
    \end{itemize}
\end{frame}

\begin{frame}{Forschungslücke}
    \textbf{Unzureichend erforscht:}
    \begin{itemize}
        \item Modulare Architektur noch nicht ausreichend adressiert
        \item Keine standardisierte Lösung für Integration verschiedener Verfahren
        \item Fehlende Infrastruktur für kapselierte Rekonstruktion in VR
    \end{itemize}
\end{frame}

\begin{frame}{Forschungsfrage}
    \vfill
    \begin{center}
        \large
        \textit{Wie gut eignet sich eine modulare, containerisierte Systemarchitektur zur\\
        Integration verschiedener Echtzeit-3D-Rekonstruktionsverfahren\\
        in eine bestehende Virtual-Reality-Umgebung?}
    \end{center}
    \vfill
\end{frame}
