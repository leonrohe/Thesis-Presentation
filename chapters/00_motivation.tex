\section{Motivation}\sectionFrame

\begin{frame}{Virtual Reality: Stark interaktiv, aber wenig realitätsnah}
    \begin{itemize}
        \item Moderne Virtual-Reality-Systeme ermöglichen \textbf{hochgradig natürliche Interaktion}
        \item Virtuelle Umgebungen sind jedoch überwiegend \textbf{künstlich modelliert}
        \item Die \textbf{reale Umgebung der Nutzenden} bleibt dabei unberücksichtigt
        \item $\Rightarrow$ Begrenzte Immersion und eingeschränkte Anwendbarkeit
    \end{itemize}
\end{frame}

\begin{frame}{Warum reale 3D-Rekonstruktion in VR wichtig ist}
    \begin{itemize}
        \item Reale Räume erhöhen die \textbf{Glaubwürdigkeit} virtueller Umgebungen
        \item Essenziell für
        \begin{itemize}
            \item kollaboratives Arbeiten
            \item simulationsbasiertes Lernen
            \item räumliche Analyse
        \end{itemize}
        \item \textbf{Spontane Nutzung} ohne manuelle Modellierung realer Szenen
    \end{itemize}
\end{frame}

\begin{frame}{Technische Hürden}
    \begin{itemize}
        \item Klassische Photogrammetrie liefert \textbf{hochqualitative Rekonstruktionen}
        \item Für interaktive VR-Anwendungen jedoch \textbf{nicht echtzeitfähig}
        \item Moderne monokulare Verfahren sind \textbf{hardware-effizient}
        \item \textbf{Fehlende modulare Integration} erschwert den praktischen Einsatz
    \end{itemize}
\end{frame}
