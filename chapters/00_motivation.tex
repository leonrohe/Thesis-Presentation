\section{Motivation}\sectionFrame

\begin{frame}{Warum 3D-Rekonstruktion in VR?}
    \begin{itemize}
        \item Moderne VR-Systeme ermöglichen natürliche Interaktion in virtuellen Räumen
        \item Viele VR-Erfahrungen basieren auf künstlich erstellten Szenen
        \item Für kollaboratives Arbeiten, simulationsbasiertes Lernen und räumliche Analysen: enge Beziehung zwischen digitaler und realer Welt erforderlich
        \item Glaubwürdige virtuelle Räume leiten sich aus realen Szenen ab
    \end{itemize}
\end{frame}

\begin{frame}{Praktische Vorteile der Rekonstruktion}
    \begin{itemize}
        \item Räume können spontan erfasst und gemeinsam analysiert werden
        \item Kollaborative Aufgaben ohne aufwendige manuelle Modellierungsprozesse
        \item Szenen-Erfassung ermöglicht dynamische VR-Anwendungen
        \item Grundbaustein für nächste Generation interaktiver VR-Systeme
    \end{itemize}
\end{frame}

\begin{frame}{Problem: Klassische Photogrammetrie}
    \begin{itemize}
        \item Hochauflösende Resultate
        \item \textbf{Aber:} umfangreiche Bildaufnahmen erforderlich
        \item \textbf{Aber:} lange Verarbeitungszeiten
        \item \textbf{Ergebnis:} ungeeignet für dynamische und interaktive VR-Anwendungen
    \end{itemize}
\end{frame}

\begin{frame}{Lösungsansatz: Monokulare Echtzeit-Rekonstruktion}
    \begin{itemize}
        \item Fortlaufende RGB-Videodaten ermöglichen schrittweise Geometrie-Rekonstruktion
        \item Monokulare Verfahren spielen zentrale Rolle
        \item \textbf{Vorteil:} keine zusätzliche Hardware erforderlich
        \item \textbf{Vorteil:} direkte Nutzung von Frontkameras moderner VR-Headsets
    \end{itemize}
\end{frame}

\begin{frame}{Zentrale Herausforderung}
    \begin{itemize}
        \item Monokulare Rekonstruktionsverfahren grundsätzlich für VR geeignet
        \item \textbf{Problem:} praktische Einbindung in bestehende Anwendungen kaum etabliert
        \item Viele Modelle entwickelt in isolierten Forschungskontexten
        \item Heterogene Laufzeitumgebungen, Frameworks und Datenpipelines
        \item Nicht auf Einsatz in interaktiven Systemen ausgelegt
    \end{itemize}
\end{frame}

\begin{frame}{Was VR-Anwendungen benötigen}
    \begin{itemize}
        \item Stabiler und klar strukturierter Datenfluss
        \item Kontinuierlich wachsende Modelle verlässlich bereitstellen
        \item Nahtlose Integration ins Frontend
        \item Vergleichbarer Betrieb mehrerer Rekonstruktionsverfahren
    \end{itemize}
\end{frame}

\begin{frame}{Forschungslücke}
    \begin{itemize}
        \item Modulare Architektur noch nicht ausreichend adressiert
        \item Keine Lösung für standardisierte Integration verschiedener Verfahren
        \item Fehlende Infrastruktur für kapselte und vereinheitlichte Rekonstruktion in VR
    \end{itemize}
\end{frame}

\begin{frame}{Forschungsfrage}
    \begin{center}
        \Large \textit{Wie gut eignet sich eine modulare, containerisierte Systemarchitektur zur Integration verschiedener Echtzeit-3D-Rekonstruktionsverfahren in eine bestehende Virtual-Reality-Umgebung?}
    \end{center}
\end{frame}
