\section{Fazit}\sectionFrame

\begin{frame}{Echtzeitfähigkeit und GPU-Engpässe}
    \begin{columns}[T]
        \column{0.5\textwidth}
        \textbf{Volumetrische Verfahren}
        \begin{itemize}
            \itemsep0.4em
            \item Durchsatz: 0,30–0,62 fps
            \item GPU-Auslastung: 43–56\%
            \item Detaillierte Oberflächenqualität
            \item Geschlossene Meshes
        \end{itemize}
        
        \column{0.5\textwidth}
        \textbf{SLAM-basierte Verfahren}
        \begin{itemize}
            \itemsep0.4em
            \item Durchsatz: 0,03–0,13 fps
            \item GPU-Auslastung: 98–100\%
            \item VRAM-Bottleneck: 7,9–8,0 GB
            \item GTX 1070 Ti limitiert
        \end{itemize}
    \end{columns}
    
    \vfill
    \textbf{Latenzcharakteristik}
    \begin{itemize}
        \itemsep0.4em
        \item Inferenz dominiert: 63–90\% der Gesamtlatenz
        \item Netzwerk + Rendering: <1000ms (konstanter Overhead)
    \end{itemize}
\end{frame}

\begin{frame}{Architektur: Containerisierung und Modularität}
    \textbf{Erfolgreiche Integration heterogener Verfahren} \hfill \textit{\small 4 Modelle, konfliktäre PyTorch-Versionen}
    \begin{itemize}
        \itemsep0.4em
        \item 47–66 Zeilen Dockerfile, 205–401 Zeilen Worker-Code
        \item Integrationsaufwand: 4–8 Stunden pro Modell
        \item BaseReconstructionModel kapselt WebSocket-Logik vollständig
    \end{itemize}
    
    \vspace{0.8em}
    
    \textbf{Standardisierte Schnittstellen} \hfill \textit{\small Framework-Heterogenität abstrahiert}
    \begin{itemize}
        \itemsep0.4em
        \item Router-Overhead: 20\% CPU, 100 MB RAM
        \item Fragment-Größen: 0,82–1,84 MB
        \item Parallelbetrieb ohne Konflikte
        \item Fan-Out stabil (2 Szenen, 4 Clients, 6h)
    \end{itemize}
\end{frame}

\begin{frame}{Rekonstruktionsqualität und SDK-Limitationen}
    \begin{columns}[T]
        \column{0.5\textwidth}
        \textbf{Synthetische Szenen}
        \begin{itemize}
            \itemsep0.4em
            \item Volumetrisch stabil (F: 0,41–0,68)
            \item VisFusion beste Performance in V3
            \item SLAM-Verfahren mit Drift
            \item SLAM3R: V1→V3 ↓31\%
        \end{itemize}
        
        \column{0.5\textwidth}
        \textbf{Reale VR-Szenen}
        \begin{itemize}
            \itemsep0.4em
            \item Leistungsumkehr: SLAM überlegen
            \item Volumetrisch fragmentiert
            \item MASt3R-SLAM trotz niedrigster fps Sieger
        \end{itemize}
    \end{columns}
    
    \vfill
    \begin{tcolorbox}[colback=yellow!5, colframe=gray!70, rounded corners, boxsep=0.5em]
        \textbf{Ursachen:}
        \begin{itemize}
            \item Meta Quest 3 fehlende Frame-Timestamps → zeitliche Inkonsistenzen (ms) → TSDF-Artefakte.
            \item SLAM-Verfahren kompensieren durch Pose-Optimierung. 
            \item SDK-Limitationen primär limitierend, nicht Algorithmen.
        \end{itemize}
    \end{tcolorbox}
\end{frame}

\begin{frame}{Beantwortung der Forschungsfrage}
    \begin{center}
        \textit{Wie gut eignet sich eine modulare, containerisierte Architektur zur Integration heterogener Rekonstruktionsverfahren in VR?        }
    \end{center}
    
    \vspace{0.6em}
    
    \textbf{Antwort: Sehr gut – empirisch validiert durch 4 Verfahren}
    \begin{itemize}
        \itemsep0.4em
        \item Standardisierte Schnittstellen abstrahieren Framework-Heterogenität erfolgreich
        \item Asynchrone Streaming-Architektur reduziert Kopplungsgrad signifikant
        \item Containerisierung ermöglicht flexible Hardware-Nutzung und Isolation
        \item Kontextspezifische Verfahrenswahl ermöglicht (volumetrisch vs. SLAM)
        \item Va.Si.Li-Lab: Additive Erweiterung ohne invasiven Eingriff
    \end{itemize}
    
    \vfill
    \begin{tcolorbox}[colback=yellow!5, colframe=gray!70, rounded corners, boxsep=0.5em]
        \small \textbf{Wissenschaftlich:} Systematische Vergleiche identifizieren SDK-bedingte Limitationen
        
        \textbf{Praktisch:} Wiederverwendbare Forschungsinfrastruktur mit minimalem Integrationsaufwand (4–8h)
    \end{tcolorbox}
\end{frame}

\begin{frame}{Limitationen}
    \begin{columns}[T]
        \column{0.5\textwidth}
        \textbf{Performance und Hardware}
        \begin{itemize}
            \itemsep0.4em
            \item Durchsatz 0,03–0,62 fps unzureichend
            \item Keine flüssige Echtzeitvisualisierung
            \item GPU-Engpass: 8 GB VRAM
            \item Parallelbetrieb limitiert
        \end{itemize}
        
        \column{0.5\textwidth}
        \textbf{Architektur und SDK}
        \begin{itemize}
            \itemsep0.4em
            \item Kein universales Modell
            \item Container-Neustarts → Datenverlust
            \item Manuelle SLAM-Skalierung erforderlich
            \item Meta Quest 3: Fehlende Timestamps
        \end{itemize}
    \end{columns}
    
    \vfill
    \begin{tcolorbox}[colback=yellow!5, colframe=gray!70, rounded corners, boxsep=0.5em]
        \textbf{Trade-off:} Geschwindigkeit vs. Robustheit – Kontextabhängige Verfahrenswahl notwendig
    \end{tcolorbox}
\end{frame}

\begin{frame}{Zukünftige Arbeiten}
    \textbf{Architekturoptimierungen}
    \begin{itemize}
        \itemsep0.4em
        \item PostgreSQL + PostGIS: Langfristige Zustandsverwaltung
        \item GPU-Ressourcen-Management: Dynamische VRAM-Allokation
        \item Adaptive Modell-Selektion: Automatische Verfahrenswahl
        \item Kompression und inkrementelles Meshing
        \item Automatische Kalibrierung: Selbständige Skalierung und Ausrichtung
    \end{itemize}
    
    \vspace{0.4em}
    
    \textbf{Neue Repräsentationsverfahren}
    \begin{itemize}
        \itemsep0.4em
        \item Gaussian Splatting mit neuer Visualisierungs-Pipeline
        \item NeRF-basierte Verfahren: Instant-NGP, TensoRF
        \item Hybride Ansätze: Volumetrische Robustheit + SLAM-Pose-Optimierung
    \end{itemize}
    
    \vspace{0.4em}
    
    \textbf{Wissenschaftliche Perspektiven}
    \begin{itemize}
        \itemsep0.4em
        \item Standard VR-Rekonstruktions-Benchmark über Hardware und Szenen
        \item User Studies: Wahrnehmung von Artefakten, akzeptable Update-Frequenzen
        \item Multi-View-Fusion: Multi-Headset-Rekonstruktion für kollaborative Szenarien
    \end{itemize}
\end{frame}