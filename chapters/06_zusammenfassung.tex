\section{Fazit}\sectionFrame

\begin{frame}{Echtzeitfähigkeit und GPU-Engpässe}
    \begin{itemize}
        \item \textbf{Volumetrische Verfahren} (NeuralRecon, VisFusion): Höchster Durchsatz
        \begin{itemize}
            \item 0,30–0,62 fps, moderate GPU-Auslastung (43–56\%)
            \item Detaillierte Oberflächenqualität, geschlossene Meshes
        \end{itemize}
        \item \textbf{SLAM-basierte Verfahren} (MASt3R-SLAM, SLAM3R): Geringerer Durchsatz
        \begin{itemize}
            \item 0,03–0,13 fps, vollständige GPU-Auslastung (98–100\%)
            \item VRAM-Bottleneck: 7,9–8,0 GB (GTX 1070 Ti limitiert)
        \end{itemize}
        \item \textbf{Infraenzzeit dominiert:} 63–90\% der Gesamtlatenz
        \item \textbf{Netzwerk + Rendering:} Nur \textless1000ms, minimal overhead
    \end{itemize}
\end{frame}

\begin{frame}{Containerisierung und Modularität}
    \begin{itemize}
        \item \textbf{Erfolgreiche Integration:} 4 heterogene Verfahren mit konfliktären PyTorch-Versionen
        \item \textbf{Geringer Integrationsaufwand:}
        \begin{itemize}
            \item 47–66 Zeilen Dockerfile pro Modell
            \item 205–401 Zeilen Worker-Code (durchschn. 4–8 Stunden)
        \end{itemize}
        \item \textbf{Standardisierte Schnittstellen:} Framework-Heterogenität vollständig abstrahiert
        \item \textbf{BaseReconstructionModel:} Kapselt WebSocket-Logik, nur Inference-Methode nötig
        \item \textbf{Router-Overhead:} 20\% CPU, 100 MB RAM bei 0,82–1,84 MB Fragment-Größen
        \item \textbf{Isolation:} Paralleler Betrieb ohne Konflikte
    \end{itemize}
\end{frame}

\begin{frame}{Rekonstruktionsqualität in synthetischen Szenen}
    \begin{itemize}
        \item \textbf{Volumetrische Verfahren} stabil über Szenen-Komplexität
        \begin{itemize}
            \item F-Scores: 0,41–0,68
            \item VisFusion beste Performance in V3 (F=0,62)
            \item NeuralRecon führend in V2 (F=0,68)
        \end{itemize}
        \item \textbf{SLAM-basierte Verfahren} zeigen Drift bei Komplexität
        \begin{itemize}
            \item SLAM3R: V1 (F=0,61) → V3 (F=0,42) ↓ 31\%
            \item MASt3R-SLAM stabil, aber durchgehend unter volumetrischen (F=0,45–0,50)
        \end{itemize}
        \item \textbf{Trade-off:} Artefaktfreiheit (volumetrisch) vs.\ Detailtreue (SLAM)
    \end{itemize}
\end{frame}

\begin{frame}{Leistungsumkehr in realen VR-Szenen}
    \begin{itemize}
        \item \textbf{Kritische Beobachtung:} Volumetrische Verfahren fragmentiert, großflächige Lücken
        \item \textbf{SLAM-Verfahren überlegen:} Deutlich höhere Vollständigkeit und Detailtreue
        \item \textbf{Ursache:} Meta Quest 3 fehlende Frame-Timestamps
        \begin{itemize}
            \item Zeitliche Inkonsistenzen (ms) → räumliche Artefakte in TSDF-Fusion
            \item SLAM-Verfahren kompensieren durch interne Pose-Optimierung
        \end{itemize}
        \item \textbf{Sieger:} MASt3R-SLAM trotz niedrigster Inferenzgeschwindigkeit
        \item \textbf{Schlussfolgerung:} SDK-Limitationen, nicht Algorithmen, primär limitierend
    \end{itemize}
\end{frame}

\begin{frame}{Eignung für VR-Anwendungen}
    \begin{itemize}
        \item \textbf{Hardware-Kompatibilität:} Monokulare Verfahren prinzipiell geeignet
        \begin{itemize}
            \item Abhängig von SDK-Qualität (Timestamps, Tracking-Präzision)
        \end{itemize}
        \item \textbf{Kontextabhängige Verfahrenswahl:}
        \begin{itemize}
            \item Synthetische Umgebungen → volumetrisch
            \item Consumer-VR-Headsets → SLAM
        \end{itemize}
        \item \textbf{Latenz-Anforderungen:} Durchsätze (0,03–0,62 fps) zu niedrig für flüssige Aktualisierung
        \item \textbf{Multi-User-Stabilität:} Fan-Out funktioniert fehlerfrei (2 Szenen, 4 Clients, 6h)
        \item \textbf{Integration:} Va.Si.Li-Lab erfolgreich additive Erweiterung, kein invasiver Eingriff
    \end{itemize}
\end{frame}

\begin{frame}{Beantwortung der Forschungsfrage}
    \begin{itemize}
        \item \textbf{Frage:} Wie gut eignet sich eine modulare, containerisierte Architektur zur Integration heterogener Rekonstruktionsverfahren in VR?
        \item \textbf{Antwort:} \textbf{Gut bis sehr gut} – empirisch validiert durch 4 Verfahren
        \begin{itemize}
            \item Standardisierte Schnittstellen abstrahieren Framework-Heterogenität erfolgreich
            \item Containerisierung ermöglicht flexible Hardware-Nutzung
            \item Asynchrone Streaming-Architektur reduziert Kopplungsgrad signifikant
        \end{itemize}
        \item \textbf{Wissenschaftlicher Mehrwert:} Systematische Vergleiche ermöglichen Identifikation SDK-bedingter Limitationen
        \item \textbf{Praktisches Resümee:} Wiederverwendbare Forschungsinfrastruktur mit minimalem Integrationsaufwand
    \end{itemize}
\end{frame}

\begin{frame}{Limitationen und Trade-offs}
    \begin{itemize}
        \item \textbf{Algorithmen:} Kein universales Modell (Geschwindigkeit vs.\ Robustheit)
        \item \textbf{Hardware:} GPU-Engpass (GTX 1070 Ti, 8 GB VRAM) limitiert Parallelbetrieb
        \item \textbf{Durchsatz:} 0,03–0,62 fps unzureichend für echtzeitflüssige Visualisierung
        \item \textbf{SDK-Limitationen:} Meta Quest 3 Frame-Timestamps beeinflussen volumetrische Verfahren erheblich
        \item \textbf{Fehlende Persistenz:} Container-Neustarts → Datenverlust, keine Zustandsverwaltung
        \item \textbf{Manuelle Skalierung:} SLAM-Rekonstruktionen erfordern Alignment (interne Koordinatensysteme)
    \end{itemize}
\end{frame}

\begin{frame}{Zukünftige Arbeiten: Architektur}
    \begin{itemize}
        \item \textbf{Persistente Zustandsverwaltung:} PostgreSQL + PostGIS für langfristige Speicherung
        \item \textbf{Adaptive Modell-Selektion:} Automatische Verfahrenswahl basierend auf Szenen-Charakteristika
        \item \textbf{GPU-Ressourcen-Management:} Dynamische VRAM-Allokation, Batch-Processing
        \item \textbf{Kompression und Inkrementelles Meshing:} Mesh-Differencing zur Netzwerk-Optimierung
        \item \textbf{Automatische Kalibrierung:} Selbständige Skalierung und Ausrichtung SLAM-basierter Rekonstruktionen
    \end{itemize}
\end{frame}

\begin{frame}{Zukünftige Arbeiten: Modelle und Forschung}
    \begin{itemize}
        \item \textbf{Gaussian Splatting:} Fotorealistische Rekonstruktionen mit neuer Visualisierungs-Pipeline
        \item \textbf{NeRF-basierte Verfahren:} Instant-NGP, TensoRF (Trainingszeiten weiterhin Herausforderung)
        \item \textbf{Hybride Ansätze:} Volumetrische Robustheit + SLAM-Pose-Optimierung
        \item \textbf{Benchmark-Suite:} Standard VR-Rekonstruktions-Benchmark über Hardware und Szenen
        \item \textbf{User Studies:} Wahrnehmung von Artefakten, akzeptable Update-Frequenzen
        \item \textbf{Multi-View-Fusion:} Multi-Headset-Rekonstruktion für kollaborative Szenarien
    \end{itemize}
\end{frame}

\begin{frame}{Abschließende Bewertung}
    \begin{itemize}
        \item RTReconstruct demonstriert Realisierbarkeit einer \textbf{deployment-freundlichen, modularen Pipeline}
        \item Containerisierte Architektur ermöglicht \textbf{systematische Vergleiche unter identischen Bedingungen}
        \item Identifikation SDK-bedingter Limitationen \textbf{(Meta Quest 3)} zeigt Mehrwert kontrollierter Evaluation
        \item \textbf{Praktischer Integrationsaufwand gering:} 4–8h pro Modell, wartbare Infrastruktur
        \item \textbf{Trade-off-Analyse:} Kontextspezifische Verfahrenswahl ermöglicht (volumetrisch vs.\ SLAM)
        \item \textbf{Potenzial:} Standard-Infrastruktur für zukünftige 3D-Sensing-Forschung in VR-Umgebungen
    \end{itemize}
\end{frame}
