\section{Grundlagen}\sectionFrame

\begin{frame}{Virtual Reality}
    \begin{itemize}
        \item Inside-Out-Tracking: Kameras im Headset erfassen Umgebung für Positionsbestimmung
        \item Position \& Orientierung relativ zu definiertem Referenzpunkt berechnet
        \item Passthrough-Funktionalität: Frontkameras zeigen Live-Bild der realen Umgebung
        \item Ermöglicht Mixed-Reality-Anwendungen: Virtuelle Inhalte in realer Umgebung eingebettet
        \item Kontinuierliche RGB-Videodaten direkt von HMD-Kameras verfügbar
        \item Ideal für monokulare 3D-Rekonstruktion ohne zusätzliche Hardware
    \end{itemize}
\end{frame}

\begin{frame}{3D-Rekonstruktion: Volumetrische Verfahren}
    \begin{itemize}
        \item TSDF-Repräsentation: Truncated Signed Distance Field
        \item Diskrete Gitter von Voxeln mit Distanzinformationen
        \item Positive Distanz = Bereich vor Oberfläche
        \item Negative Distanz = Bereich hinter Oberfläche
        \item Marching Cubes-Algorithmus für Mesh-Generierung
        \item Schrittweise Fusion mehrerer Ansichten → global konsistent
        \item GPU-beschleunigte, effiziente Echtzeit-Updates
    \end{itemize}
\end{frame}

\begin{frame}{3D-Rekonstruktion: SLAM-Verfahren}
    \begin{itemize}
        \item Klassisch: Visuelle Merkmals-Verfolgung über mehrere Frames
        \item Gleichzeitige Optimierung von Kameraposen und 3D-Punkten
        \item Lernbasiert: Hybride Verfahren mit neuronalen Netzen
        \item Robuster gegen schwierige Beleuchtung \& texturarme Szenen
        \item Monokulare Verfahren ideal für VR-Headsets ohne dedizierte Tiefensensoren
    \end{itemize}
\end{frame}

\begin{frame}{Systemtechnologien: Client-Server-Architektur}
    \begin{itemize}
        \item Clients: Datenerzeugung \& Visualisierung
        \item Server: Rechenintensive Verarbeitung auf GPU
        \item Vorteilhaft für mobile VR mit begrenzte Rechenleistung
        \item Bidirektionale, persistente Verbindung erforderlich
        \item Asynchrone Kommunikation für Echtzeitanwendungen
    \end{itemize}
\end{frame}

\begin{frame}{Systemtechnologien: WebSocket-Protokoll}
    \begin{itemize}
        \item Vollständig bidirektional über persistente TCP-Verbindung
        \item Server kann jederzeit Daten an Clients senden
        \item Gegenüber HTTP-Polling:
        \begin{itemize}
            \item Drastisch reduzierte Latenz
            \item Minimierter Netzwerk-Overhead
            \item Ideal für Streaming mit kontinuierlichem Datenfluss
        \end{itemize}
        \item Dauerhafte Verbindung ohne Verbindungsaufbau-Kosten
    \end{itemize}
\end{frame}

\begin{frame}{Systemtechnologien: Containerisierung}
    \begin{itemize}
        \item Container: Isolierte Laufzeitumgebungen mit allen Abhängigkeiten
        \item Docker: Verbreitetste Implementierung
        \item Docker Compose: Orchestrierung mehrerer Container über yml-Datei
        \item NVIDIA Container Toolkit: GPU-Zugriff innerhalb von Containern
        \item Ressourcenschonender als virtuelle Maschinen
        \item Konsistente Ausführung unabhängig von Host-Umgebung
        \item Ideal für heterogene Rekonstruktionsverfahren mit verschiedenen GPU-Abhängigkeiten
    \end{itemize}
\end{frame}

\begin{frame}{Zielumgebung: Unity}
    \begin{itemize}
        \item Unity: Führende 3D-Engine für interaktive Anwendungen
        \item Komponentenorientierte Architektur mit GameObjects
        \item Spezialisierte XR-Subsysteme für VR
        \item Head-Tracking, stereoskopisches Rendering, HMD-Integration
        \item C\#-Scripting-API für flexible Logik \& Netzwerk
        \item Unterstützung moderner Grafik-APIs: OpenGL, Vulkan, DirectX
        \item Plattformübergreifend: Mobile \& hochauflösende VR
    \end{itemize}
\end{frame}

\begin{frame}{Zielumgebung: Va.Si.Li-Lab}
    \begin{itemize}
        \item VR-Lab for Simulation-based Learning - Goethe-Universität Frankfurt
        \item Mehrbenutzerfähige VR-Plattform für simulationsbasierte Lernszenarien
        \item Fokus: Pädagogisch-professionelle Handlungssituationen
        \item Ubiq-Framework für Netzwerkkommunikation \& Synchronisation
        \item Unterstützt verschiedene Endgeräte \& Remote-Teilnahme
        \item Modulare Architektur ermöglicht Integration neuer Funktionalitäten
        \item Ideale Zielplattform für RTReconstruct-Integration
    \end{itemize}
\end{frame}