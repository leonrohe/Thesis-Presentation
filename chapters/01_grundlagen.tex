\section{Grundlagen}\sectionFrame

\begin{frame}{Grundlagen: Virtual Reality \& monokulare Rekonstruktion}
    \textbf{VR-Hardware heute:}
    \begin{itemize}
        \item HMDs wie Meta Quest 3 mit integrierten RGB-Kameras
        \item Kein Tiefensensor $\Rightarrow$ klassische RGB-D-Verfahren nicht einsetzbar
    \end{itemize}
    
    \vspace{0.8em}
    \textbf{RTReconstruct-Ziel:}
    \begin{itemize}
        \item Tiefe aus Bildsequenzen rekonstruieren (SLAM, volumetrische Verfahren)
        \item Ergebnisse in laufende VR-Anwendung streamen
        \item Mehrere Rekonstruktionsmodelle parallel austauschbar
    \end{itemize}
\end{frame}

\begin{frame}{Grundlagen: 3D-Rekonstruktion}
    \begin{itemize}
        \item Prozess: 2D-Bilder $\rightarrow$ 3D-Geometrie
        \item \textbf{Monokulare Rekonstruktion}: Tiefe aus Kamerabewegung + Bildsequenzen
        \item Zwei zentrale Paradigmen:
        \begin{itemize}
            \setlength{\itemsep}{\lineskip}
            \item \textbf{Volumetrisch} (z.\,B. NeuralRecon): Benötigt externe Kameraposen
            \item \textbf{SLAM-basiert} (z.\,B. MASt3R-SLAM): Schätzt Posen selbst
        \end{itemize}
        \item Diese Heterogenität ist zentral für das System-Design
    \end{itemize}
\end{frame}

\begin{frame}{Grundlagen: 3D-Repräsentationsformate}
    \begin{columns}[T]
        \column{0.5\textwidth}
        \textbf{Punktwolken}
        \begin{itemize}
            \item Ungeordnete 3D-Punkte
            \item Speichereffizient
            \item Schnelle Visualisierung
        \end{itemize}
        
        \column{0.5\textwidth}
        \textbf{Meshes}
        \begin{itemize}
            \item Explizite Oberflächengeometrie
            \item GLB-Format (binär, kompakt)
            \item Besser für Kollisionen
        \end{itemize}
    \end{columns}
    \vfill
    \only<2>{
        \begin{center}
            \textbf{RTReconstruct muss beide Formate handhaben!}    
        \end{center}
    }
\end{frame}

\begin{frame}{Grundlagen: Zielumgebung Va.Si.Li-Lab}
    \begin{itemize}
        \item \textbf{Multi-User VR-Plattform}: Basis-Technologie = Unity + Ubiq-Framework
        \item Designed für \textit{kollaborative, simulationsbasierte Lernszenarien}
        \item Bereits vorhanden: Tracking, Netzwerk-Sync, Szenenlogik
        \item \vspace{0.5em}
        \textbf{RTReconstruct als Modul}:
        \begin{itemize}
            \setlength{\itemsep}{\lineskip}
            \item Nutzt bestehende \texttt{Tracking-System} für Kameradaten
            \item Erhält Rekonstruktionsergebnisse von Backend
            \item Visualisiert in gemeinsamem 3D-Raum
        \end{itemize}
    \end{itemize}
\end{frame}